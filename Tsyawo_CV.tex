\documentclass[12pt,letterpaper]{article}

\usepackage{hyperref}
\usepackage[margin=1in]{geometry}
\usepackage{multicol}

% Comment the following lines to use the default Computer Modern font
% instead of the Palatino font provided by the mathpazo package.
% Remove the 'osf' bit if you don't like the old style figures.
% \usepackage[T1]{fontenc}
% \usepackage[sc,osf]{mathpazo}

% Set your name here
\def\name{Emmanuel Selorm Tsyawo}

% The following metadata will show up in the PDF properties
\hypersetup{
  colorlinks = true,
  urlcolor = blue,
  pdfauthor = {\name},
  pdfkeywords = {economics}
  pdftitle = {\name: Curriculum Vitae},
  pdfsubject = {Curriculum Vitae},
  pdfpagemode = UseNone
}

% \geometry{
%   body={6.5in, 8.5in},
%   left=1.0in,
%   top=1.25in
% }

% Customize page headers
\pagestyle{myheadings}
\markright{\name}
\thispagestyle{empty}

% Custom section fonts
\usepackage{sectsty}
\sectionfont{\bfseries\scshape\Large}
\subsectionfont{\rmfamily\mdseries\scshape\large}

% Other possible font commands include:
% \ttfamily for teletype,
% \sffamily for sans serif,
% \bfseries for bold,
% \scshape for small caps,
% \normalsize, \large, \Large, \LARGE sizes.

% Don't indent paragraphs.
\setlength\parindent{0em}

% Make lists without bullets
\renewenvironment{itemize}{
  \begin{list}{}{
    \setlength{\leftmargin}{1.5em}
  }
}{
  \end{list}
}

\begin{document}

\begin{minipage}{\linewidth}
  \begin{flushright}
    Updated \today
  \end{flushright}
\end{minipage}

\vspace{10pt}

% Place name at left
%{\huge \name}

\centerline{\huge \bf \name}

\vspace{0.25in}

\begin{minipage}{0.65\linewidth}
  Department of Economics \\
  Temple University \\
  Ritter Annex 873 \\
  1301 Cecil B. Moore Ave \\
  Philadelphia, PA 19122
\end{minipage}
\begin{minipage}{0.5\linewidth}
  \begin{tabular}{r}
    \href{mailto:estsyawo@temple.edu}{\tt estsyawo@temple.edu} \\
    \href{https://estsyawo.github.io}{\tt https://estsyawo.github.io} \\
    \emph{Citizenship:} Ghana\\
    \emph{Graduate Coordinator:} \\
    Ms. Linda Wyatt, \\
    +1 215 204 6638, {\color{blue} ldwyatt@temple.edu}
    
  \end{tabular}
\end{minipage}

\vspace{10pt}
\rule{\linewidth}{0.4pt}


%\section*{Academic Appointments}
%\begin{itemize}
%  \item Adjunct Professor, Temple University, Department of Economics, 2018-
%\end{itemize}

\section*{Education}

\begin{itemize}
  \item Ph.D. Candidate, Economics, Temple University, 2018- (\textit{Expected}: May 2020)

  \begin{itemize}
    \item Dissertation: ``Estimating R\&D Interaction Structures and Spillover Effects''
    \item Committee:   \href{https://bcallaway11.github.io/index.html}
    {Brantly Callaway} (Chair), \href{https://astro.temple.edu/~rytchkov/}{Oleg Rytchkov}, \href{https://liberalarts.temple.edu/academics/faculty/swanson-charles}{Charles Swanson}, \href{https://liberalarts.temple.edu/academics/faculty/bean-austin}{Austin Bean} 
  \end{itemize}

	\item Ph.D. Student, Economics, Temple University, 2014-

  \item B.A., Economics and French (First Class), University of Ghana, Legon, 2013
\end{itemize}

\section*{Research Interests}
\begin{itemize}
  \item Microeconometrics and Empirical Industrial Organisation
\end{itemize}

\section*{Research Skills}
\begin{itemize}
	\item Bayesian Econometrics, Machine Learning, Statistical programming
\end{itemize}

\section*{Working Papers} % Indicate JMP?
\begin{enumerate}
	\item  \href{https://estsyawo.github.io/Tsyawo_JMP.pdf#}{R\&D spillover effects on firm innovation - Estimating the spatial matrix from panel data}, (Job Market Paper)
	
	\textbf{Abstract}: Firms’ research and development (R\&D) efforts are known to have spillover effects on other
firms’ outcomes, e.g., innovation. Quantifying R\&D spillover effects on innovation requires a
	spatial matrix that characterises the strength of connectivity between firms. Estimates can be
	biased if the spatial matrix is misspecified. This paper proposes a parsimonious approach to
	estimating parameters and the spatial matrix from panel data, when the spatial matrix is partly
	or fully unknown, in order to quantify R\&D spillovers on innovation. The approach is applicable
	to single index models, and it allows asymmetry and time-variation in the spatial matrix.
	The paper establishes consistency and asymptotic normality of the MLE under conditional
	independence and conditional strong-mixing assumptions on the outcome variable. On firm
	innovation, we find positive spillover and private effects of R\&D. We provide evidence of timevariation and asymmetry in the interaction structure between firms and find that geographic
	proximity and product market proximity are relevant. Moreover, the strength of connectivity
	between firms is not limited to often-assumed notions of closeness; it is also linked to past R\&D
and patenting behaviour of firms.
	
	\item \href{https://papers.ssrn.com/sol3/papers.cfm?abstract_id=3394012}{Clustered Covariate Regression}, (with Abdul-Nasah Soale, under review) - \textit{Presented at the Mid-West Econometrics Group Conference, October 2019}
	
	\textbf{Abstract}: This paper introduces an estimator for a general class of models under rank deficiency arising from high dimensionality, multicollinearity, or both. Our approach obtains a projection matrix that projects a high-dimensional (potentially growing $ p >> n $) parameter vector into a reduced consistently estimable one. We show consistency and asymptotic normality of the estimator. Recovering the high-dimensional parameter vector using the projection matrix leaves precision unaffected. We employ a sequential estimation algorithm that, at once, obtains parameter estimates and the projection matrix. Our Monte Carlo simulations demonstrate a high approximative ability of high-dimensional parameters, improved precision, and reduced bias even under multicollinearity. In our empirical application, we find that firms on average generate positive R\&D spillovers on firm productivity though these are dominated by private returns to R\&D.
	
	
	\item \href{https://papers.ssrn.com/sol3/papers.cfm?abstract_id=3048658}{Bayesian Distribution Regression}, (with Weige Huang)
	
	\textbf{Abstract}: This paper introduces a Bayesian version of distribution regression that enables inference on estimated distributions, quantiles, distributional effects, among other functionals of interest. Our estimators come in three categories: the non-asymptotic, semi-asymptotic, and asymptotic. To conduct simultaneous inference on a function of any estimator, we introduce asymmetric and symmetric Bayesian confidence bands. Inference on point estimates is conducted via posterior intervals. The Bayesian asymptotic theory we develop extends the foregoing to gains in computational time and tractability of posterior distributions. Monte Carlo simulations conducted illustrate good performance of our estimators. We apply our estimators to evaluate the impact of institutional ownership on firm innovation. 
	
	\item \href{https://papers.ssrn.com/sol3/papers.cfm?abstract_id=3194286}{Recovering distributions for the estimation of treatment
		effects under partly unobserved treatment with repeated
		cross-sections}
	
	\textbf{Abstract}: This paper develops an approach to estimating quantile treatment effects among other distribution-based measures of treatment effects when treatment status is unobserved in some periods by using a maximum likelihood-based finite mixture approach that combines the multinoulli and the asymmetric Laplace distributions (ALD). We show that by treating the unobserved treatment status as latent and using the Expectation-Maximisation algorithm to recover unobserved treatment statuses and distributions, employing quantile differences-indifferences (QDID), changes-in-changes (CIC) among other estimators of treatment effects is straightforward. To evaluate the small sample performance of our model, we employ Monte Carlo simulations and find that the model generally outperforms naive ones that substitute available proxies for unobserved treatment.
\end{enumerate}

\section*{Work in Progress}
\begin{enumerate}
	\item Unobserved heterogeneity in two-part models (with Brantly Callaway)
	\item A parsimonious approach to estimating the SAR spatial matrix from panel data
	\item Estimating the SAR spatial matrix by Clustered Covariate Regression
\end{enumerate}

\section*{Conferences}
\begin{itemize}
	\item October 2019: Presenter, Mid-West Econometrics Group Conference, Columbus, Ohio
\end{itemize}

\section*{Teaching}
\begin{enumerate}
	\item Temple University
	\begin{itemize}
		\item Instructor, Principles of Macroeconomics, Fall 2018, Spring 2019, Fall 2019
		\item Instructor, Principles of Microeconomics, Summer I (2015 \& 2016), Fall 2017
		\item Instructor, Intermediate Microeconomics, Summer I, 2018
		\item Teaching Assistant for Prof. Michael Bonnano, Fall 2014 - Spring 2016
		\item Teaching Assistant for Prof. Michael Leeds, Fall 2016 - Spring 2017
	\end{itemize}
	\item University of Ghana, Legon
	\begin{itemize}
		\item Teaching Assistant for \href{https://www.ug.edu.gh/economics/staff/abel-fumey}{Dr. Abel Fumey}, 2013-2014
	\end{itemize}
\end{enumerate}


\section*{Software}
\begin{itemize}
  \item \textbf{R packages}: \texttt{bayesdistreg} [\texttt{\href{https://CRAN.R-project.org/package=bayesdistreg}{CRAN}}] [\texttt{\href{https://estsyawo.github.io/bayesdistreg/}{Website}}] [\texttt{\href{https://github.com/estsyawo/bayesdistreg}{Github}}] (with Weige Huang), \texttt{cluscov} [\texttt{\href{https://CRAN.R-project.org/package=cluscov}{CRAN}}] [\texttt{\href{https://estsyawo.github.io/cluscov/}{Website}}] [\texttt{\href{https://github.com/estsyawo/cluscov}{GitHub}}](with Abdul-Nasah Soale), \texttt{RpacSPD} [\texttt{\href{https://estsyawo.github.io/RpacSPD/}{Website}}] [\texttt{\href{https://github.com/estsyawo/RpacSPD}{GitHub}}]

  \item \textbf{C projects}: \texttt{metricsC} [\texttt{\href{https://github.com/estsyawo/metricsC}{GitHub}}], \texttt{RdotC} [\texttt{\href{https://estsyawo.github.io/RdotC/index.html}{Website}}] [\texttt{\href{https://github.com/estsyawo/RdotC}{GitHub}}], 
  
  \item \textbf{Fortran projects}: \texttt{metricsFortran} [\texttt{\href{https://github.com/estsyawo/metricsFortran}{GitHub}}]
  \item Matlab,  Stata, Octave, Python, \LaTeX, Microsoft Office
\end{itemize}

\section*{Languages}
Ewe (mother tongue), English (fluent, official language of Ghana), French (fluent, DALF C2), Spanish (fluent, DELE B2), German (intermediate),  Portuguese (intermediate), Twi%, Latin (good reading level)

\section*{Honors and Fellowships}
\begin{enumerate}
  \item Research Fellowship, Department of Economics, Summer 2019
  \item Research Fellowship, Department of Economics, Summer 2017
\end{enumerate}
%\newpage
\section*{References}
\vspace{-.15in}
\begin{multicols}{2}
	
	\textbf{Prof. Brantly Callaway} \\
	(Dissertation Advisor)\\
	Department of Economics, \\
	University of Mississippi \\
	\hfill {Phone: +1 662 915 6942}\\
	\hfill{E-mail: {\color{blue}bmcallaw@olemiss.edu}}
	
	\vspace{.2in}
	
	\textbf{Prof. Oleg Rytchkov}\\
	(Dissertation Committee Member)\\
	Department of Finance, TU\\
	\hfill Phone: +1 215 204 4146\\
	\hfill{E-mail: {\color{blue}rytchkov@temple.edu}}\\
%		\vspace{.8in}

	%\vspace{.1in}	
\textbf{Prof. Doug Webber} \\
Department of Economics, TU \\
\hfill{Phone: +1 215 204 5025} \\
\hfill{E-mail: {\color{blue}douglas.webber@temple.edu}}\\
%\vspace{.3in}

		\columnbreak
		
		\textbf{Prof. Charles Swanson} \\
	(Dissertation Committee Member)\\
	Department of Economics, TU \\
	\hfill {Phone: +1 215 204 8168}\\
	\hfill{E-mail: {\color{blue}swansonc@temple.edu}}\\
	
	\vspace{.2in}
	\textbf{Prof. Austin Bean} \\
	(Dissertation Committee Member)\\
	Department of Economics, TU \\
	\hfill {Phone: +1 215 204 1680}\\
	\hfill{E-mail: {\color{blue}austin.bean@temple.edu}}\\	


	
	\vspace{.2in}
	\textbf{Prof. Michael Bognanno} \\
	Department of Economics, TU \\
	\hfill {Phone: +215 204 1680}\\
	\hfill{E-mail: {\color{blue}bognanno@temple.edu}}\\	
%	
%	\vspace{.3in}
%	\textbf{Prof. Michael Leeds} \\
%	Department of Economics, TU \\
%	\hfill{Phone: (215) 204-8030} \\
%	\hfill{E-mail: {\color{blue}mleeds@temple.edu}}\\		
\end{multicols}

\end{document}
